In conclusion, we are able to to rediscover the true dynamics of a Lorenz system and the nonlinear pendulu through our autoencoder SINDy implementation. 
In doing so, we validate our results against the findings of \textcite{Champion_2019}, achieving similar results. 
Our findings indicate that 

Nevertheless, different potential improvements might be introduced to aid the convergence and robustness of the model. 
As the aim of our experiments was to display the autoencoder SINDy framework, we set a constant learning rate similar to \textcite{Champion_2019}. 
Nevertheless, \textsc{ADAM} is frequently used with decreasing learning rates through learning rate schedulers to achieve higher precision, as it often is unstable at high learning rates. 

Moreover, alternative optimization algorithms to \textsc{ADAM} might provide fruitful in improving the performance of the framework. 
One reason for this is that \textsc{ADAM} relies on first - and second order - moments to determine its gradient direction, relying on its memory of older gradients. 
However, pairing this with \textit{sequential threshold least squares} means that we periodically change the loss landscape by masking away certain SINDy coefficients. 
This approach is potentially inharmonious with \textsc{ADAM}'s moment-based approach as the momentum is not representative of the updated loss landscape. 
In addition, by introducing regularization into the loss formulation, we rely on \textsc{ADAM}'s gradient based methods, differentiating the non-differentiable $L_1$-regularization loss.
Although this yields sufficient performance in the aforementioned experiments, other optimization methods that rectify these potential challenges might yield improved performance. 

In fact, \textcite{SR3} have proposed an alternative optimization algorithm to sequential threshold least squares named SR3. 
SR3 is a sparse regression method, applying \textit{relaxation} to the original loss formulation and relying on proximal gradients for determining optimzal descent directions on non-smooth objectives, promising computational efficiency, higher accuracy, faster convergence rates, and greater flexibility compared to traditional methods. 
SR3 has already successfully been demonstrated for regular SINDy \cite{SR3_SINDy}, yielding strong performance. 
Hence, it might be constructive to further explore the SR3 algorihtm, and to extend it to the Autoencoder SINDy framework. 

Another challenge with the Autoencoder SINDy formulation is the introduction of increased hyperparameters through the loss weights, resulting in having to balance reconstruction, regularization, and the two dynamical losses. 
\textcite{Champion_2019} provides some empirically successful methods to set the hyperparameters, but it is uncertain if a more complex tuning stage can yield improvements to the training algorithm. 

%%%% Discuss graph neural networks, perhaps alternative SINDy models. This remains to be done
Alternative SINDy formulations also exist, like implicit SINDy \textcite{} ... . However, as seen by the experiments, the computational load of Autoencoder SINDy can become quite significant, ... . Hence, it might be interesting to investigate how one can incoroporate similar ideas to implicit SINDy into the Autoencoder framework. 

Scientific machine learning seems to be increasingly developing towards reliable and interpretable models. 
With the advent of improved sparse regularization methods in addition to existing dynamical discovery frameworks like Autoencoders, the Autoencoder SINDy framework provides a blueprint to discovering parsimonious governing equations of physical systems, thus aiding the interpretability of such systems. 